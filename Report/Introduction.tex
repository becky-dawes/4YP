\chapter{Introduction}

\section{Background}

Sequence prediction has possible applications in many different areas. We come across it every day whilst using the search engine Google when it helpfully suggests possible search topics based on what we have already typed. This concept of "autocomplete" is something which we take for granted in this situation, but which could be extended to many different uses. A rather different application is that of a therapeutic system. Individuals with motor disorders may find some sort of text prediction service useful when using phones or computers (much like "autocorrect"). 

This project was originally based on the work done by IBM on Watson, a system capable of answering questions. It is best known for its performance on the gameshow Jeopardy!, where it won against human opponents. Watson was trained using a variety of datasources, but it was not connected to the internet during the game. Although it did occasionally struggle to understand the concept of the questions and was unable to buzz in early, Watson's reaction times were much faster than those of the humans (about 8ms to activate the buzzer after receiving an electronic signal compared to the tenths of a second taken for the humans to perceive a light signal) and so it beat them on most of the questions.

The Sequence Memoizer (developed in 2011) is a Bayesian nonparametric model for discrete sequence data. Its design is discussed in more detail in Chapter \ref{chap:seqMem}. 

\todo[inline]{More work on Background}

\todo[inline]{Section on Watson}

\section{The Purpose of the Project}

\todo[inline]{Purpose of project - specification and goals}

The purpose of this project was to design and build a stochastic 

What is the problem?

Why study it?

Main objectives

Overview of report structure